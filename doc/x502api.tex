\documentclass[12pt,a4paper,titlepage]{report}
\usepackage[utf8]{inputenc}
\usepackage[english, russian]{babel}

\usepackage{indentfirst} %для отступа в первом абзаце
\usepackage{longtable} %таблицы с переносом на несколько страниц
%для создание оглавления в pdf и гиперссылок
\usepackage[pdftex, unicode, final, colorlinks,linkcolor=blue,urlcolor=blue,bookmarksnumbered]{hyperref}
\usepackage{array}
\usepackage[pdftex]{graphicx}

\usepackage{cmap} % чтобы работал поиск по PDF
\usepackage{doxygen}
\usepackage{textcomp}
%\usepackage{floatflt}

\hypersetup{
  pdftitle = {Руководство программиста для модулей L502 и E502},
  pdfauthor = {Борисов А.В.}
}


\oddsidemargin=5.0mm     % +10mm -25.4mm
\evensidemargin=-10mm   % -10mm
\textwidth=165mm
\topmargin=-10mm
\textheight=250mm

\emergencystretch=75pt

\setcounter{secnumdepth}{3}
\setcounter{tocdepth}{4}

\begin{document}
  \begin{titlepage}
    \topmargin=10mm
    \vspace{45mm}

    % Название класса устройств
    \begin{flushright}
    {\bfseries\scshape\Large Современные устройства сбора данных}
    \end{flushright}

    \rule{160mm}{2mm}                % Горизонтальная черта

    % ------------------------------------------------
    % название конкретного устройства
    % ------------------------------------------------
    \begin{flushright}
    {\bfseries\scshape\Huge L502/E502}
    \end{flushright}

    % ------------------------------------------------
    % тип документа (имя)
    % ------------------------------------------------
    \begin{flushright}
    {\bfseries\scshape\Large Руководство программиста}
    \end{flushright}

    \vspace{155mm}

    % ------------------------------------------------
    % Логотип фирмы + дата публикации + версия док-ии
    % ------------------------------------------------
    %\begin{floatingfigure}[l]{30mm}
    %\includegraphics[width=40mm]{../../SharedImages/LCardLogo.pdf}
    %\end{floatingfigure}

	%\begin{floatingfigure}[l]{30mm}
	%\includegraphics[width=30mm]{../SharedImages/LCardLogo.png} 
	%\includegraphics[width=40mm]{LCardLogo.png} 
	%\end{floatingfigure}
	
    \noindent
    \begin{flushright}
    {\itshape\footnotesize Ревизия 1.1.3 \\ Июль 2015}
    \end{flushright}

  \end{titlepage}
  
  
  \vspace{5mm}
  \begin{flushleft}
  \textbf{Автор руководства:} \\
  Борисов А.В.
  \end{flushleft}
  
  \vspace{5mm}
  
  % ------------------------------------------------
  % Адрес фирмы
  % ------------------------------------------------
  
  \label{lcard_address}
  \begin{flushleft}
  \bfseries\large ООО "Л КАРД" \\
  {\mdseries\normalsize 117105, г. Москва, Варшавское ш., д. 5, корп. 4, стр. 2 }
  \end{flushleft}
  
  \vspace{2mm}
  
  % ------------------------------------------------
  % Телефоны
  % ------------------------------------------------
  \begin{flushleft}
  тел.: 	(495) 785-95-25 \\
  факс: 	(495) 785-95-14 \\
  \end{flushleft}
  
  \vspace{2mm}
  
  % ------------------------------------------------
  % Доступ в интернет
  % ------------------------------------------------
  \begin{flushleft}
  \bfseries\large Адреса в Интернет:  \\
  \mdseries\normalsize \href{http://www.lcard.ru}{www.lcard.ru} \\
  %\mdseries\normalsize \href{ftp://ftp.lcard.ru/pub}{ftp.lcard.ru/pub} \\
  \end{flushleft}
  
  \vspace{2mm}
  
  % ------------------------------------------------
  % Электронная почта
  % ------------------------------------------------
  \begin{flushleft}
  \bfseries\large E-Mail: \\
  \mdseries\normalsize Отдел продаж:           sale@lcard.ru \\ 
  \mdseries\normalsize Техническая поддержка:  support@lcard.ru \\
  \mdseries\normalsize Отдел кадров:           job@lcard.ru \\
  \mdseries\normalsize Общие вопросы:          lcard@lcard.ru \\
  \mdseries\normalsize Отдел производства:     pro@lcard.ru \\
  \end{flushleft}
  
  \vspace{2mm}
  
  % ------------------------------------------------
  % Офисы по стране
  % ------------------------------------------------
  \begin{flushleft}
  \bfseries\large Представители в регионах:  \\
  \mdseries\normalsize Украина: ХОЛИТ Дэйта Системс,\href{http://www.holit.com.ua}{www.holit.com.ua}, (044) 241-6754 \\
  \mdseries\normalsize Санкт-Петербург: Autex Spb Ltd., \href{http://www.autex.spb.ru}{www.autex.spb.ru}, (812) 567-7202 \\
  \mdseries\normalsize Новосибирск: Сектор-Т, \href{http://www.sector-t.ru}{www.sector-t.ru}, (383-2) 396-592 \\
  \mdseries\normalsize Казань: ООО 'Шатл', shuttle@kai.ru, (8432) 38-1600  \\
  \mdseries\normalsize Екатеринбург: Aвеон, aveon@aveon.ru, +7(343) 381-75-75  \\
  \mdseries\normalsize Пенза: НПП Технолинк, \href{http://www.tl.ru/ru/departments/industry/}{http://www.tl.ru/ru/departments/industry} (8412) 49-10-59  \\
  \end{flushleft}
  
  % ---------------------------------------------------------------
  % (!!!) Дублирование имени физический интерфейс устройства, 
  % приведенного на титульной странице
  % ---------------------------------------------------------------
  
  \vspace{65mm}
  
  {\itshape Модули L502/E502} \copyright~Copyrigh 2015, ООО "Л Кард". Все права защищены.
  \newpage
  
  
  %----------------------------------------------------
  % Ревизии документа 
  %---------------------------------------------------------
  \begin{longtable}{|m{0.18\linewidth}|m{0.18\linewidth}|m{0.54\linewidth}|}
  \caption{Ревизии текущего документа}\\\hline
  \textbf{Ревизия} & \textbf{Дата} & \textbf{Описание}\\\hline
  1.0.0            & 27.06.2012    & Первая ревизия данного документа \\\hline
  1.0.1            & 22.11.2012    & Добавлено описание использования библиотеки с программами на C\# и в LabView, добавлено описание установки для ОС Linux, а также описание функций для циклического вывода \\\hline
  1.0.2            & 20.02.2013    & Добавлена ссылка на исходные коды SDK. Описание установки пакетов для Linux вынесено в отдельный документ \\\hline
  1.0.3            & 16.02.2015    & Исправлена последовательность шагов для работы с модулем при синхронном потоковом выводе. Добавлено примечание о передачи массивов в качестве выходных параметров в LabView \\\hline
  1.1.0            & 02.06.2015    & Описание изменено в соответствии с изменениями, внесенными в библиотеку для поддержки модуля E502 (введение общих и специализированных функций). Включено краткое описание различий модулей с программной стороны. Добавлены отдельные главы, описывающие настройку модуля при работе по Ethernet и поиск модулей в локальной сети. \\\hline
  1.1.1            & 06.07.2015    & Добавлено описание возможности ожидания завершения установки циклического сигнала, добавленной в версии 1.1.2 библиотеки. В разделе отличий модулей при описании наличия ARM-контроллера в E502 указан путь для скачивания обновлений прошивки с рекомендацией обновления. \\\hline
  1.1.2            & 10.07.2015    & Добавлено описание нового алгоритма расчета максимального размера циклического сигнала для E502 с прошивкой ARM 1.0.3 и выше \\\hline
  1.1.3            & 28.07.2015    & Добавлено описание функций X502\_SetExtRefFreqValue() и X502\_GetRefFreqValue() \\\hline
  \end{longtable}
  \newpage  
  
  \tableofcontents
  

\chapter{О чем этот документ}
\input{sect_about}
\chapter{Установка и подключение библиотеки к проекту}
\input{sect_setup}
\chapter{Общий подход к работе с библиотекой}
\input{sect_gen_descr}

  
\chapter{Константы, типы данных и функции библиотеки}
\input{constants}

\input{types}
%\input{struct_l502_cbr_coef.tex}
%\input{struct_l502_cbr.tex}
%\input{struct_l502_info.tex}
\section{Функции}
\input{funcs_hnd}
\input{funcs_open}
\input{funcs_devrec}
\input{funcs_config}
\input{funcs_async}
\input{funcs_streams}
\input{funcs_eth_config}
\input{funcs_eth_svc_browse}
\input{funcs_dsp}
\input{funcs_flash}
\input{funcs_misc}



\end{document}
